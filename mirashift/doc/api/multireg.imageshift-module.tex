%
% API Documentation for ImageShift
% Module multireg.imageshift
%
% Generated by epydoc 2.1
% [Thu Dec 15 12:26:52 2005]
%

%%%%%%%%%%%%%%%%%%%%%%%%%%%%%%%%%%%%%%%%%%%%%%%%%%%%%%%%%%%%%%%%%%%%%%%%%%%
%%                          Module Description                           %%
%%%%%%%%%%%%%%%%%%%%%%%%%%%%%%%%%%%%%%%%%%%%%%%%%%%%%%%%%%%%%%%%%%%%%%%%%%%

    \index{multireg.imageshift \textit{(module)}|(}
\section{Module multireg.imageshift}

    \label{multireg:imageshift}
This module implements a multi-scale transform based method for 
determining the offset between 2 images.

The original algorithm was implemented by ESO for their automated image 
registration in the ESI Imaging Survey image processing pipeline, as 
described by B. Vandame (2002, SPIE 4847, p. 123). The developers are 
grateful for the cooperation of the ESO developers in providing a copy of 
some of the EIS software for review during the development of this 
package.

Version 0.1 (Initial version) - WJH (3-Dec-2004) Version 0.2 - WJH 
(15-Dec-2005)


%%%%%%%%%%%%%%%%%%%%%%%%%%%%%%%%%%%%%%%%%%%%%%%%%%%%%%%%%%%%%%%%%%%%%%%%%%%
%%                               Variables                               %%
%%%%%%%%%%%%%%%%%%%%%%%%%%%%%%%%%%%%%%%%%%%%%%%%%%%%%%%%%%%%%%%%%%%%%%%%%%%

  \subsection{Variables}

\begin{longtable}{|p{.30\textwidth}|p{.62\textwidth}|l}
\cline{1-2}
\cline{1-2} \centering \textbf{Name} & \centering \textbf{Description}& \\
\cline{1-2}
\endhead\cline{1-2}\multicolumn{3}{r}{\small\textit{continued on next page}}\\\endfoot\cline{1-2}
\endlastfoot\raggedright \_\-\_\-v\-e\-r\-s\-i\-o\-n\-\_\-\_\- & \raggedright \textbf{Value:} 
{\tt '\-0\-.\-2\-~\-(\-1\-5\-~\-D\-e\-c\-e\-m\-b\-e\-r\-~\-2\-0\-0\-5\-)\-'\-}            \textit{(type=\texttt{str})}&\\
\cline{1-2}
\end{longtable}

    \index{multireg.imageshift \textit{(module)}!ImageShift \textit{(class)}|(}

%%%%%%%%%%%%%%%%%%%%%%%%%%%%%%%%%%%%%%%%%%%%%%%%%%%%%%%%%%%%%%%%%%%%%%%%%%%
%%                           Class Description                           %%
%%%%%%%%%%%%%%%%%%%%%%%%%%%%%%%%%%%%%%%%%%%%%%%%%%%%%%%%%%%%%%%%%%%%%%%%%%%

\subsection{Class ImageShift}

    \label{multireg:imageshift:ImageShift}
\begin{alltt}
The ImageShift class serves as the primary interface for
computing offsets between images using the multi-scale
wavelet transform. 

=================================
DEVELOPMENT NOTE:  
This code may eventually need to support fitting to a reference
image or coordinate list from a reference frame.
=================================

Syntax:
    ImageShift(input,output='shifts',reference=None, coeffs='header',
               scale=2,form='linear')
                
Inputs:
    input       - list of input filenames
    output      - name of output shiftfile
                    if nothing is specified, defaults to 'shifts'
    reference   - user-specified reference image
                    if None (default), first image from 
                    input will be used
    coeffs      - parameter used to specify source (if any) of 
                    distortion model to be applied to input images.
                    This corresponds directly to MultiDrizzle
                    'coeffs' and PyDrizzle 'idckey' parameters.
    scale       - number of wavelet transforms to apply: 4 (default)
    form        - form of interpolation kernel to use with wavelet
                    transforms: spline or linear (default)
Methods:
    .run(verbose=False,min\_match=10):
        --{\textgreater} Computes shifts and writes them to output.
        radius  - object matching threshold in pixels
        min\_match   - only compute shift if image has 
                        at least 'min\_match' objects 
        verbose - print computed shifts interactively

    .writeShiftFile(shift\_list = None, output=None):
        --{\textgreater} Writes out shifts to output file.
        shift\_list  - list of computed shifts for each image in form of:
                (filename,((xshift,yshift),rotation,(scale,xscale,yscale))
                    If no shift\_list is provided, it writes out values
                    in 'self.shifts'.
        output      - name of output shift file
                    If none is provided, defaults to name specified
                    when class was initialized.\end{alltt}


%%%%%%%%%%%%%%%%%%%%%%%%%%%%%%%%%%%%%%%%%%%%%%%%%%%%%%%%%%%%%%%%%%%%%%%%%%%
%%                                Methods                                %%
%%%%%%%%%%%%%%%%%%%%%%%%%%%%%%%%%%%%%%%%%%%%%%%%%%%%%%%%%%%%%%%%%%%%%%%%%%%

  \subsubsection{Methods}

    \label{multireg:imageshift:ImageShift:__init__}
    \index{multireg.imageshift \textit{(module)}!ImageShift \textit{(class)}!\_\_init\_\_ \textit{(method)}}
    \vspace{0.5ex}

    \begin{boxedminipage}{\textwidth}

    \raggedright \textbf{\_\_init\_\_}(\textit{self}, \textit{input}, \textit{output}=\texttt{'\-s\-h\-i\-f\-t\-s\-'\-}, \textit{reference}=\texttt{N\-o\-n\-e\-}, \textit{coeffs}=\texttt{'\-h\-e\-a\-d\-e\-r\-'\-}, \textit{scale}=\texttt{2\-}, \textit{form}=\texttt{'\-l\-i\-n\-e\-a\-r\-'\-}, \textit{median}=\texttt{1\-})

    \end{boxedminipage}

    \label{multireg:imageshift:ImageShift:getPositionArrays}
    \index{multireg.imageshift \textit{(module)}!ImageShift \textit{(class)}!getPositionArrays \textit{(method)}}
    \vspace{0.5ex}

    \begin{boxedminipage}{\textwidth}

    \raggedright \textbf{getPositionArrays}(\textit{self}, \textit{objlist})

    \vspace{-1.5ex}

    \rule{\textwidth}{0.5\fboxrule}
    Return detected positions as numarray arrays

    \vspace{1ex}

    \end{boxedminipage}

    \label{multireg:imageshift:ImageShift:run}
    \index{multireg.imageshift \textit{(module)}!ImageShift \textit{(class)}!run \textit{(method)}}
    \vspace{0.5ex}

    \begin{boxedminipage}{\textwidth}

    \raggedright \textbf{run}(\textit{self}, \textit{verbose}=\texttt{F\-a\-l\-s\-e\-}, \textit{min\_match}=\texttt{1\-0\-}, \textit{overwrite}=\texttt{T\-r\-u\-e\-})

    \vspace{-1.5ex}

    \rule{\textwidth}{0.5\fboxrule}
\begin{alltt}
Perform the matching between the position lists, then perform
a generalized linear fit to find the shifts.  These shifts
then get written out to a shiftfile.

The 'verbose' parameter turns on/off output of compute shifts to
STDOUT, with the default being turned off (quiet mode).

=================================
DEVELOPMENT NOTE:
    This needs to be expanded to support iteration over all
    resolution scales, and for all input images.
=================================

ImageMatch Algorithm:
Xiaolong Dai, Siamak Khorram, "A Feature-based Image Registration
Algorithm Using Improved Chain-code Representation Combined with
Invariant Moments", IEEE Trans. Geo. and Remote Sensing,
Vol 37, No. 5, 2351-2362, September 1999.\end{alltt}

    \vspace{1ex}

    \end{boxedminipage}

    \label{multireg:imageshift:ImageShift:writeCoordFile}
    \index{multireg.imageshift \textit{(module)}!ImageShift \textit{(class)}!writeCoordFile \textit{(method)}}
    \vspace{0.5ex}

    \begin{boxedminipage}{\textwidth}

    \raggedright \textbf{writeCoordFile}(\textit{self}, \textit{imagename}, \textit{scale}, \textit{objlist})

    \vspace{-1.5ex}

    \rule{\textwidth}{0.5\fboxrule}
    Write out object list to the coordinate file.

    \vspace{1ex}

    \end{boxedminipage}

    \label{multireg:imageshift:ImageShift:writeCoords}
    \index{multireg.imageshift \textit{(module)}!ImageShift \textit{(class)}!writeCoords \textit{(method)}}
    \vspace{0.5ex}

    \begin{boxedminipage}{\textwidth}

    \raggedright \textbf{writeCoords}(\textit{self}, \textit{scale}=\texttt{0\-})

    \vspace{-1.5ex}

    \rule{\textwidth}{0.5\fboxrule}
    Write out coordinate files for all input observations.

    \vspace{1ex}

    \end{boxedminipage}

    \label{multireg:imageshift:ImageShift:writeShiftFile}
    \index{multireg.imageshift \textit{(module)}!ImageShift \textit{(class)}!writeShiftFile \textit{(method)}}
    \vspace{0.5ex}

    \begin{boxedminipage}{\textwidth}

    \raggedright \textbf{writeShiftFile}(\textit{self}, \textit{shift\_list}=\texttt{N\-o\-n\-e\-}, \textit{output}=\texttt{N\-o\-n\-e\-})

    \vspace{-1.5ex}

    \rule{\textwidth}{0.5\fboxrule}
    Write out a shiftfile.

    \vspace{1ex}

    \end{boxedminipage}

    \index{multireg.imageshift \textit{(module)}!ImageShift \textit{(class)}|)}
    \index{multireg.imageshift \textit{(module)}|)}
