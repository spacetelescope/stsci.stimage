%
% API Documentation for ImageShift
% Module multireg.chipwavelets
%
% Generated by epydoc 2.1
% [Thu Dec 15 12:26:52 2005]
%

%%%%%%%%%%%%%%%%%%%%%%%%%%%%%%%%%%%%%%%%%%%%%%%%%%%%%%%%%%%%%%%%%%%%%%%%%%%
%%                          Module Description                           %%
%%%%%%%%%%%%%%%%%%%%%%%%%%%%%%%%%%%%%%%%%%%%%%%%%%%%%%%%%%%%%%%%%%%%%%%%%%%

    \index{multireg.chipwavelets \textit{(module)}|(}
\section{Module multireg.chipwavelets}

    \label{multireg:chipwavelets}

%%%%%%%%%%%%%%%%%%%%%%%%%%%%%%%%%%%%%%%%%%%%%%%%%%%%%%%%%%%%%%%%%%%%%%%%%%%
%%                               Functions                               %%
%%%%%%%%%%%%%%%%%%%%%%%%%%%%%%%%%%%%%%%%%%%%%%%%%%%%%%%%%%%%%%%%%%%%%%%%%%%

  \subsection{Functions}

    \label{multireg:chipwavelets:compute_ccode_matrix}
    \index{multireg.chipwavelets \textit{(module)}!compute\_ccode\_matrix \textit{(function)}}
    \vspace{0.5ex}

    \begin{boxedminipage}{\textwidth}

    \raggedright \textbf{compute\_ccode\_matrix}(\textit{img\_ccode}, \textit{ref\_ccode}, \textit{Lccode}=\texttt{0\-.\-1\-0\-0\-0\-0\-0\-0\-0\-0\-0\-0\-0\-0\-0\-0\-0\-1\-}, \textit{Tccode}=\texttt{0\-.\-8\-0\-0\-0\-0\-0\-0\-0\-0\-0\-0\-0\-0\-0\-0\-0\-4\-})

    \end{boxedminipage}

    \label{multireg:chipwavelets:convert_1d}
    \index{multireg.chipwavelets \textit{(module)}!convert\_1d \textit{(function)}}
    \vspace{0.5ex}

    \begin{boxedminipage}{\textwidth}

    \raggedright \textbf{convert\_1d}(\textit{index}, \textit{shape})

    \end{boxedminipage}

    \label{multireg:chipwavelets:find_cog}
    \index{multireg.chipwavelets \textit{(module)}!find\_cog \textit{(function)}}
    \vspace{0.5ex}

    \begin{boxedminipage}{\textwidth}

    \raggedright \textbf{find\_cog}(\textit{array}, \textit{xpos}, \textit{ypos}, \textit{limit})

    \end{boxedminipage}

    \label{multireg:chipwavelets:perform_ImageMatch}
    \index{multireg.chipwavelets \textit{(module)}!perform\_ImageMatch \textit{(function)}}
    \vspace{0.5ex}

    \begin{boxedminipage}{\textwidth}

    \raggedright \textbf{perform\_ImageMatch}(\textit{imgobs}, \textit{refobs}, \textit{scale}, \textit{T\_ccode}=\texttt{0\-.\-8\-0\-0\-0\-0\-0\-0\-0\-0\-0\-0\-0\-0\-0\-0\-0\-4\-}, \textit{T\_moment}=\texttt{0\-.\-1\-0\-0\-0\-0\-0\-0\-0\-0\-0\-0\-0\-0\-0\-0\-0\-1\-})

    \vspace{-1.5ex}

    \rule{\textwidth}{0.5\fboxrule}
    Implement the Steps 2-6 of the ImageMatch algorithm on an input 
    Observation object, img\_obs, and a reference Observation, ref\_obs.

    T\_ccode: threshold for chain-code matching T\_moment: threshold for 
    invariant moment matching

    \vspace{1ex}

    \end{boxedminipage}


%%%%%%%%%%%%%%%%%%%%%%%%%%%%%%%%%%%%%%%%%%%%%%%%%%%%%%%%%%%%%%%%%%%%%%%%%%%
%%                               Variables                               %%
%%%%%%%%%%%%%%%%%%%%%%%%%%%%%%%%%%%%%%%%%%%%%%%%%%%%%%%%%%%%%%%%%%%%%%%%%%%

  \subsection{Variables}

\begin{longtable}{|p{.30\textwidth}|p{.62\textwidth}|l}
\cline{1-2}
\cline{1-2} \centering \textbf{Name} & \centering \textbf{Description}& \\
\cline{1-2}
\endhead\cline{1-2}\multicolumn{3}{r}{\small\textit{continued on next page}}\\\endfoot\cline{1-2}
\endlastfoot\raggedright \_\-\_\-v\-e\-r\-s\-i\-o\-n\-\_\-\_\- & \raggedright \textbf{Value:} 
{\tt '\-0\-.\-3\-.\-0\-~\-(\-2\-7\--\-S\-e\-p\-t\--\-2\-0\-0\-5\-)\-'\-}            \textit{(type=\texttt{str})}&\\
\cline{1-2}
\end{longtable}

    \index{multireg.chipwavelets \textit{(module)}!Chip \textit{(class)}|(}

%%%%%%%%%%%%%%%%%%%%%%%%%%%%%%%%%%%%%%%%%%%%%%%%%%%%%%%%%%%%%%%%%%%%%%%%%%%
%%                           Class Description                           %%
%%%%%%%%%%%%%%%%%%%%%%%%%%%%%%%%%%%%%%%%%%%%%%%%%%%%%%%%%%%%%%%%%%%%%%%%%%%

\subsection{Class Chip}

    \label{multireg:chipwavelets:Chip}
\begin{alltt}
This class keeps track of all the wavelet transformations
for a chip, and performs the object finding on those transformed
images.

Input:
    imagename   - full image name complete with extension
                    such as 'test\_flt.fits[sci,1]'.
    imagearray  - numarray object containing the science data for image
    offset      - zero-point offset of chip relative to final output frame
    pyasn       - PyDrizzle object relating image to output frame
                    if None, perform no distortion correction in positions
    scale       - Number of wavelet transformations to apply to image
    form        - form of wavelet interpolation: spline or linear (default)
    photzpt     - photometric zero-point appropriate for this chip
    photflam    - photometric conversion factor to convert counts to flux
Methods:
    getPositions(scale=0)       
        - returns list of undistorted positions for objects 
            identified at specified wavelet transformation scale
    getRawPositions(scale=0)    
        - returns list of original positions for objects identified
            at specified wavelet transformation scale\end{alltt}


%%%%%%%%%%%%%%%%%%%%%%%%%%%%%%%%%%%%%%%%%%%%%%%%%%%%%%%%%%%%%%%%%%%%%%%%%%%
%%                                Methods                                %%
%%%%%%%%%%%%%%%%%%%%%%%%%%%%%%%%%%%%%%%%%%%%%%%%%%%%%%%%%%%%%%%%%%%%%%%%%%%

  \subsubsection{Methods}

    \label{multireg:chipwavelets:Chip:__init__}
    \index{multireg.chipwavelets \textit{(module)}!Chip \textit{(class)}!\_\_init\_\_ \textit{(method)}}
    \vspace{0.5ex}

    \begin{boxedminipage}{\textwidth}

    \raggedright \textbf{\_\_init\_\_}(\textit{self}, \textit{exposure}, \textit{keep\_wavelets}=\texttt{F\-a\-l\-s\-e\-}, \textit{scale}=\texttt{2\-}, \textit{form}=\texttt{'\-l\-i\-n\-e\-a\-r\-'\-}, \textit{median}=\texttt{1\-})

    \end{boxedminipage}

    \label{multireg:chipwavelets:Chip:addDelta}
    \index{multireg.chipwavelets \textit{(module)}!Chip \textit{(class)}!addDelta \textit{(method)}}
    \vspace{0.5ex}

    \begin{boxedminipage}{\textwidth}

    \raggedright \textbf{addDelta}(\textit{self}, \textit{delta})

    \end{boxedminipage}

    \label{multireg:chipwavelets:Chip:cleanWavelets}
    \index{multireg.chipwavelets \textit{(module)}!Chip \textit{(class)}!cleanWavelets \textit{(method)}}
    \vspace{0.5ex}

    \begin{boxedminipage}{\textwidth}

    \raggedright \textbf{cleanWavelets}(\textit{self})

    \end{boxedminipage}

    \label{multireg:chipwavelets:Chip:computeRange}
    \index{multireg.chipwavelets \textit{(module)}!Chip \textit{(class)}!computeRange \textit{(method)}}
    \vspace{0.5ex}

    \begin{boxedminipage}{\textwidth}

    \raggedright \textbf{computeRange}(\textit{self})

    \end{boxedminipage}

    \label{multireg:chipwavelets:Chip:getFluxes}
    \index{multireg.chipwavelets \textit{(module)}!Chip \textit{(class)}!getFluxes \textit{(method)}}
    \vspace{0.5ex}

    \begin{boxedminipage}{\textwidth}

    \raggedright \textbf{getFluxes}(\textit{self}, \textit{scale}=\texttt{0\-}, \textit{units}=\texttt{'\-m\-a\-g\-'\-})

    \vspace{-1.5ex}

    \rule{\textwidth}{0.5\fboxrule}
    Returns fluxes for all objects. If units='mag', fluxes will be 
    returned as magnitudes rather than electrons/counts/ADUs based on 
    photometric keywords read in for this chip.

    \vspace{1ex}

    \end{boxedminipage}

    \label{multireg:chipwavelets:Chip:getMask}
    \index{multireg.chipwavelets \textit{(module)}!Chip \textit{(class)}!getMask \textit{(method)}}
    \vspace{0.5ex}

    \begin{boxedminipage}{\textwidth}

    \raggedright \textbf{getMask}(\textit{self})

    \vspace{-1.5ex}

    \rule{\textwidth}{0.5\fboxrule}
    Return expanded version of mask for chip in output frame.

    \vspace{1ex}

    \end{boxedminipage}

    \label{multireg:chipwavelets:Chip:outputPositions}
    \index{multireg.chipwavelets \textit{(module)}!Chip \textit{(class)}!outputPositions \textit{(method)}}
    \vspace{0.5ex}

    \begin{boxedminipage}{\textwidth}

    \raggedright \textbf{outputPositions}(\textit{self}, \textit{output}, \textit{scale}=\texttt{0\-}, \textit{clean}=\texttt{T\-r\-u\-e\-})

    \vspace{-1.5ex}

    \rule{\textwidth}{0.5\fboxrule}
    Writes extracted undistorted positions to output ASCII file.

    \vspace{1ex}

    \end{boxedminipage}

    \label{multireg:chipwavelets:Chip:setDelta}
    \index{multireg.chipwavelets \textit{(module)}!Chip \textit{(class)}!setDelta \textit{(method)}}
    \vspace{0.5ex}

    \begin{boxedminipage}{\textwidth}

    \raggedright \textbf{setDelta}(\textit{self}, \textit{delta})

    \end{boxedminipage}

    \index{multireg.chipwavelets \textit{(module)}!Chip \textit{(class)}|)}
    \index{multireg.chipwavelets \textit{(module)}!Observation \textit{(class)}|(}

%%%%%%%%%%%%%%%%%%%%%%%%%%%%%%%%%%%%%%%%%%%%%%%%%%%%%%%%%%%%%%%%%%%%%%%%%%%
%%                           Class Description                           %%
%%%%%%%%%%%%%%%%%%%%%%%%%%%%%%%%%%%%%%%%%%%%%%%%%%%%%%%%%%%%%%%%%%%%%%%%%%%

\subsection{Class Observation}

    \label{multireg:chipwavelets:Observation}
\textbf{Known Subclasses:} ReferenceObs


%%%%%%%%%%%%%%%%%%%%%%%%%%%%%%%%%%%%%%%%%%%%%%%%%%%%%%%%%%%%%%%%%%%%%%%%%%%
%%                                Methods                                %%
%%%%%%%%%%%%%%%%%%%%%%%%%%%%%%%%%%%%%%%%%%%%%%%%%%%%%%%%%%%%%%%%%%%%%%%%%%%

  \subsubsection{Methods}

    \label{multireg:chipwavelets:Observation:__init__}
    \index{multireg.chipwavelets \textit{(module)}!Observation \textit{(class)}!\_\_init\_\_ \textit{(method)}}
    \vspace{0.5ex}

    \begin{boxedminipage}{\textwidth}

    \raggedright \textbf{\_\_init\_\_}(\textit{self}, \textit{name})

    \end{boxedminipage}

    \label{multireg:chipwavelets:Observation:addChip}
    \index{multireg.chipwavelets \textit{(module)}!Observation \textit{(class)}!addChip \textit{(method)}}
    \vspace{0.5ex}

    \begin{boxedminipage}{\textwidth}

    \raggedright \textbf{addChip}(\textit{self}, \textit{chip})

    \end{boxedminipage}

    \label{multireg:chipwavelets:Observation:computeFeatureMatrices}
    \index{multireg.chipwavelets \textit{(module)}!Observation \textit{(class)}!computeFeatureMatrices \textit{(method)}}
    \vspace{0.5ex}

    \begin{boxedminipage}{\textwidth}

    \raggedright \textbf{computeFeatureMatrices}(\textit{self}, \textit{refobs}, \textit{scale}=\texttt{0\-}, \textit{order}=\texttt{3\-})

    \vspace{-1.5ex}

    \rule{\textwidth}{0.5\fboxrule}
    Return the feature matrices of the input Observation relative to the 
    reference Observation; specifically, the invariant-moment distance 
    matrix, the chain-code matching matrix, and a center-of-gravity 
    matrix for each.

    \vspace{1ex}

    \end{boxedminipage}

    \label{multireg:chipwavelets:Observation:computeRange}
    \index{multireg.chipwavelets \textit{(module)}!Observation \textit{(class)}!computeRange \textit{(method)}}
    \vspace{0.5ex}

    \begin{boxedminipage}{\textwidth}

    \raggedright \textbf{computeRange}(\textit{self})

    \vspace{-1.5ex}

    \rule{\textwidth}{0.5\fboxrule}
    compute range of pixels spanned by entire observation in output 
    frame.

    This NEEDS to be run once all member chips have been added to this 
    object.

    \vspace{1ex}

    \end{boxedminipage}

    \label{multireg:chipwavelets:Observation:createMask}
    \index{multireg.chipwavelets \textit{(module)}!Observation \textit{(class)}!createMask \textit{(method)}}
    \vspace{0.5ex}

    \begin{boxedminipage}{\textwidth}

    \raggedright \textbf{createMask}(\textit{self})

    \vspace{-1.5ex}

    \rule{\textwidth}{0.5\fboxrule}
    Creates mask of entire observation in output field.

    \vspace{1ex}

    \end{boxedminipage}

    \label{multireg:chipwavelets:Observation:getChainCodes}
    \index{multireg.chipwavelets \textit{(module)}!Observation \textit{(class)}!getChainCodes \textit{(method)}}
    \vspace{0.5ex}

    \begin{boxedminipage}{\textwidth}

    \raggedright \textbf{getChainCodes}(\textit{self}, \textit{scale}=\texttt{0\-})

    \end{boxedminipage}

    \label{multireg:chipwavelets:Observation:getCoords}
    \index{multireg.chipwavelets \textit{(module)}!Observation \textit{(class)}!getCoords \textit{(method)}}
    \vspace{0.5ex}

    \begin{boxedminipage}{\textwidth}

    \raggedright \textbf{getCoords}(\textit{self}, \textit{scale}=\texttt{0\-})

    \vspace{-1.5ex}

    \rule{\textwidth}{0.5\fboxrule}
    Returns positions, weight, max, and type for all Objects detected at 
    the specified scale.

    USED ONLY BY .writeCoords() method of WaveShifts object. The 
    type/list of values returned could still be modified.

    \vspace{1ex}

    \end{boxedminipage}

    \label{multireg:chipwavelets:Observation:getFluxes}
    \index{multireg.chipwavelets \textit{(module)}!Observation \textit{(class)}!getFluxes \textit{(method)}}
    \vspace{0.5ex}

    \begin{boxedminipage}{\textwidth}

    \raggedright \textbf{getFluxes}(\textit{self}, \textit{scale}=\texttt{0\-})

    \end{boxedminipage}

    \label{multireg:chipwavelets:Observation:getMoments}
    \index{multireg.chipwavelets \textit{(module)}!Observation \textit{(class)}!getMoments \textit{(method)}}
    \vspace{0.5ex}

    \begin{boxedminipage}{\textwidth}

    \raggedright \textbf{getMoments}(\textit{self}, \textit{scale}=\texttt{0\-})

    \end{boxedminipage}

    \label{multireg:chipwavelets:Observation:getPositions}
    \index{multireg.chipwavelets \textit{(module)}!Observation \textit{(class)}!getPositions \textit{(method)}}
    \vspace{0.5ex}

    \begin{boxedminipage}{\textwidth}

    \raggedright \textbf{getPositions}(\textit{self}, \textit{scale}=\texttt{0\-})

    \end{boxedminipage}

    \label{multireg:chipwavelets:Observation:getScales}
    \index{multireg.chipwavelets \textit{(module)}!Observation \textit{(class)}!getScales \textit{(method)}}
    \vspace{0.5ex}

    \begin{boxedminipage}{\textwidth}

    \raggedright \textbf{getScales}(\textit{self})

    \end{boxedminipage}

    \label{multireg:chipwavelets:Observation:getSlices}
    \index{multireg.chipwavelets \textit{(module)}!Observation \textit{(class)}!getSlices \textit{(method)}}
    \vspace{0.5ex}

    \begin{boxedminipage}{\textwidth}

    \raggedright \textbf{getSlices}(\textit{self}, \textit{scale}=\texttt{0\-})

    \end{boxedminipage}

    \label{multireg:chipwavelets:Observation:setDelta}
    \index{multireg.chipwavelets \textit{(module)}!Observation \textit{(class)}!setDelta \textit{(method)}}
    \vspace{0.5ex}

    \begin{boxedminipage}{\textwidth}

    \raggedright \textbf{setDelta}(\textit{self}, \textit{delta})

    \vspace{-1.5ex}

    \rule{\textwidth}{0.5\fboxrule}
    Set global delta for entire observation.

    \vspace{1ex}

    \end{boxedminipage}

    \index{multireg.chipwavelets \textit{(module)}!Observation \textit{(class)}|)}
    \index{multireg.chipwavelets \textit{(module)}!ReferenceObs \textit{(class)}|(}

%%%%%%%%%%%%%%%%%%%%%%%%%%%%%%%%%%%%%%%%%%%%%%%%%%%%%%%%%%%%%%%%%%%%%%%%%%%
%%                           Class Description                           %%
%%%%%%%%%%%%%%%%%%%%%%%%%%%%%%%%%%%%%%%%%%%%%%%%%%%%%%%%%%%%%%%%%%%%%%%%%%%

\subsection{Class ReferenceObs}

    \label{multireg:chipwavelets:ReferenceObs}
\begin{tabular}{cccccc}
% Line for multireg.chipwavelets.Observation, linespec=[False]
\multicolumn{2}{r}{\settowidth{\BCL}{multireg.chipwavelets.Observation}\multirow{2}{\BCL}{multireg.chipwavelets.Observation}}
&&
  \\\cline{3-3}
  &&\multicolumn{1}{c|}{}
&&
  \\
&&\multicolumn{2}{l}{\textbf{ReferenceObs}}
\end{tabular}

Class used as reference observation for iterating the fit.


%%%%%%%%%%%%%%%%%%%%%%%%%%%%%%%%%%%%%%%%%%%%%%%%%%%%%%%%%%%%%%%%%%%%%%%%%%%
%%                                Methods                                %%
%%%%%%%%%%%%%%%%%%%%%%%%%%%%%%%%%%%%%%%%%%%%%%%%%%%%%%%%%%%%%%%%%%%%%%%%%%%

  \subsubsection{Methods}

    \label{multireg:chipwavelets:ReferenceObs:__init__}
    \index{multireg.chipwavelets \textit{(module)}!ReferenceObs \textit{(class)}!\_\_init\_\_ \textit{(method)}}
    \vspace{0.5ex}

    \begin{boxedminipage}{\textwidth}

    \raggedright \textbf{\_\_init\_\_}(\textit{self}, \textit{wcs})

      Overrides: multireg.chipwavelets.Observation.\_\_init\_\_

    \end{boxedminipage}

    \label{multireg:chipwavelets:ReferenceObs:addChip}
    \index{multireg.chipwavelets \textit{(module)}!ReferenceObs \textit{(class)}!addChip \textit{(method)}}
    \vspace{0.5ex}

    \begin{boxedminipage}{\textwidth}

    \raggedright \textbf{addChip}(\textit{self}, \textit{obs})

      Overrides: multireg.chipwavelets.Observation.addChip

    \end{boxedminipage}

    \label{multireg:chipwavelets:ReferenceObs:checkOverlap}
    \index{multireg.chipwavelets \textit{(module)}!ReferenceObs \textit{(class)}!checkOverlap \textit{(method)}}
    \vspace{0.5ex}

    \begin{boxedminipage}{\textwidth}

    \raggedright \textbf{checkOverlap}(\textit{self}, \textit{obs})

    \vspace{-1.5ex}

    \rule{\textwidth}{0.5\fboxrule}
    Determine whether this observation overlaps the current reference 
    mask.

    \vspace{1ex}

    \end{boxedminipage}

    \label{multireg:chipwavelets:ReferenceObs:overlapMask}
    \index{multireg.chipwavelets \textit{(module)}!ReferenceObs \textit{(class)}!overlapMask \textit{(method)}}
    \vspace{0.5ex}

    \begin{boxedminipage}{\textwidth}

    \raggedright \textbf{overlapMask}(\textit{self}, \textit{obs})

    \vspace{-1.5ex}

    \rule{\textwidth}{0.5\fboxrule}
    Updates mask of entire observation in output field, IF it is found to 
    overlap observations already in reference mask.

    It returns a flag denoting whether the observation overlapped or not 
    and, therefore, whether the mask was updated or not.

    It works on entire observations, rather than just chip-by-chip.

    \vspace{1ex}

    \end{boxedminipage}

    \label{multireg:chipwavelets:ReferenceObs:writeShifts}
    \index{multireg.chipwavelets \textit{(module)}!ReferenceObs \textit{(class)}!writeShifts \textit{(method)}}
    \vspace{0.5ex}

    \begin{boxedminipage}{\textwidth}

    \raggedright \textbf{writeShifts}(\textit{self}, \textit{filename})

    \end{boxedminipage}

  \textbf{Inherited from Observation:}
    computeFeatureMatrices,
    computeRange,
    createMask,
    getChainCodes,
    getCoords,
    getFluxes,
    getMoments,
    getPositions,
    getScales,
    getSlices,
    setDelta
    \index{multireg.chipwavelets \textit{(module)}!ReferenceObs \textit{(class)}|)}
    \index{multireg.chipwavelets \textit{(module)}|)}
